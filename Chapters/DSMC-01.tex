% Chapter X

\chapter{INHOMOGENEOUS GASES} % Chapter title

\label{ch:inhomogas} % For referencing the chapter elsewhere, use \autoref{ch:inhomogas} 

%----------------------------------------------------------------------------------------

Need to introduce the usefulness of the method here. References to many kinds of applications and such things. I also need to perform a literature review of all DSMC used in cold atoms physics.

Compare to molecular dynamic approaches, when is DSMC appropriate / good? When does it fail? 

Find out the knudsen number for typical cold atom conditions. Wade has numbers for stamper kern and shvatchuck.

DSMC - Birds Book \cite{Bird1994}

Evaporative Cooling and Expansion Dynamics: \cite{Wu1996, Wu1997, Wu1998}

Bosonic Collective-Mode dynamics: \cite{Jackson2001, Jackson2001b, Jackson2002, Jackson2002b, Jackson2002c}, Can't find Jackson Zaremba 2002 Laser Physics, 12, 93

Fermion Dynamics: \cite{Urban2006, Urban2007, Urban2008, Lepers2010} (see also \cite{Vignolo2002, Toschi2003, Capuzzi2004, Toschi2004})

Sympathetic Cooling: \cite{Barletta2010, Barletta2011}

Applications - Rayleigh Bernard Flow: \cite{Watanabe1994}

Spacecraft aerodynamics: \cite{Oran1998}

Chemical reactions: \cite{Anderson2003} Goldsworthy?

Microfluidics: \cite{Frangi2003}

Acoustics on Earth, Mars and Titan: \cite{Hanford2009}

Volcanic plumes on Jupiter: \cite{Zhang2004}

Read: \sout{\cite{Minguzzi2004}}, [45]

Refer to cuda section \ref{ch:cudadsmc} Should also discuss the development of this parallel implementation of the code. Compare to CPU implementations. Goldsworthy has a few references for other CUDA codes.

%----------------------------------------------------------------------------------------

\section{Collision Rates in Inhomogeneous Gases}

Overall collision rate, spatial collision rate, talk about number of cells, the occupancy of cells and the effect of inhomogeniety.

The heart of the DSMC method is to simplify the simulation of interparticle interactions in the form of two body collisions. 
The DSMC method offers some free parameters which we can optimise to balance the accuracy and efficiency, namely the number of cells, $n_c$, and the number of test particles, $N_p$.
We wish to see the effect of varying these parameters on the results of the simulation.
One of the most basic tests for the application of the DSMC method to cold atom physics is to investigate the collision rate for a thermal gas. 
We saw in section \ref{sec:colrates} that the thermally averaged collision rate per unit density for a single species atomic gas bound by the potential $\cal{U}(\mathbf{r})$ is given by
\begin{equation}
    {\tau_c}^{-1} = \frac{1}{2}\dot{n}_{0}\langle v\sigma \rangle \frac{V_{2e}}{V_e}.
\end{equation}
The simplest scenario we can use to investigate the effects of changing the free parameters is that of a thermal gas in a box. 
In this case the collision rate simply reduces to 
\begin{equation*}
    {\tau_{c,box}}^{-1} = \frac{1}{\sqrt{2}}n_0\bar{v}\sigma,
\end{equation*}
where $\bar{v}=\sqrt{8k_{B}T/\pi m}$, is the average speed of the atoms.
Include a surface plot of the collision rate as a function of cell number and test particle number.
Should make the z axis percentage error. See fig 14. in wade.

%----------------------------------------------------------------------------------------

\section{Thermalisation}

Show how things can thermalise in the required number of collision times.
Walraven section 6.4, 3 collisions to thermalise.

%----------------------------------------------------------------------------------------

\section{Evaporation}

Compare some results to those predicted by the theory of walraven and the other guy.

\section{Adiabaticity}

Have a look at squeezing the magnetic trap both diabaticaly and adiabatically.