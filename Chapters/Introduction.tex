% Chapter Introduction

\chapter{Introduction} % Chapter title

\label{ch:intro} % For referencing the chapter elsewhere, use \autoref{ch:intro} 

%----------------------------------------------------------------------------------------

\begin{flushright}{\slshape    
Begin at the beginning...\\
and go on 'till you come to the end:\\
then stop.} \\ \medskip
--- Lewis Carrol, Alice's Adventures in Wonderland
\end{flushright}

\bigskip

%----------------------------------------------------------------------------------------

There needs to be a story telling section?
One that provides context for all my work, links everything together historically mathematically and physically.
The next sections can provide any technical background required?

%----------------------------------------------------------------------------------------

I am going to need to introduce the different trapping potentials in here some where I think.

\section{Magnetic Trapping} \label{sec:intromag}

Ioffe Pritchard trap \cite{Meystre2001}
\begin{equation}
    \mathbf{B}_{\mathrm{IP}}(x,y,z) = B_0 \begin{bmatrix} 0\\ 0\\ 1 \end{bmatrix}
                      + B' \begin{bmatrix} x\\-y\\ 0 \end{bmatrix}
                      + \frac{1}{2}B'' \begin{bmatrix}-xz\\-yz\\ z^2-\frac{1}{2}\left(x^2+y^2\right) \end{bmatrix}
\end{equation}
the magnitude of this field can be approximated as the following for small position
\begin{equation}
    \left| \mathbf{B}_{\mathrm{IP}} \right| = \frac{1}{2}\left(B_{\rho}''\left(x^2+y^2\right) + B''z^2\right), \label{eq:ipmag}
\end{equation}
where 
\begin{equation*}
    B_\rho''= \frac{B'^2}{B_0} - \frac{B''}{2}. 
\end{equation*}


Quadrupole trap
\begin{equation}
    \mathbf{B}_{\mathrm{Q}}(x,y,z) = \frac{1}{2} B_z' \begin{bmatrix} x\\ y\\ -2z \end{bmatrix} \label{eq:quad_field}
\end{equation}

%----------------------------------------------------------------------------------------

\section{Collision Rates in Thermal Gases} \label{sec:collisionRates}

Overall collision rate, spatial collision rate, talk about number of cells, the occupancy of cells and the effect of inhomogeniety.
 
One of the most basic tests of the application of the DSMC method to cold atom physics is to investigate the collision rate for a thermal gas.
\marginpar{Maybe say something about the Boltzmann equation here and give some references.} 
Using the Boltzmann equation we can derive \cite{Walraven2010} the thermally averaged collision rate per unit density for a single species atomic gas bound by the potential ${\cal U}(\mathbf{r})$,
\begin{equation}
    {\tau_c}^{-1} = \frac{1}{2} n_{0}\langle v\sigma \rangle \frac{V_{2e}}{V_e}, \label{eq:collision_rate}
\end{equation}
where $n_0 = N / V_e$ is the central density of the gas, $\langle v\sigma \rangle$ is the thermally averaged product of the atomic velocity and collision cross section, \\*$V_e = \int \exp\left[-{\cal U}(\mathbf{r})/kBT\right]\, d\mathbf{r}$ is the effective volume of the gas, and \\*$V_{2e} = \int \exp\left[-2{\cal U}(\mathbf{r})/kBT\right]\, d\mathbf{r}$ the effective volume corresponding to the distribution of pairs.
\marginpar{ The effective volume, $V_e$, of an inhomogeneous gas equals the volume of a homogeneous gas with the same number of atoms and density. }
For the bulk of this work we will consider collisions in three unique trapping potentials: no trapping potential, \ie  a homogeneous gas, an Ioffe Pritchard trap(cite) and a spherical quadrupole trap(cite - is it really spherical?)\footnote{We have a more in depth discussion of magnetic trapping in appendix \ref{sec:magneticTrapping}.}. 
In table \ref{tab:collisionrates} we have derived the expressions for the effective volume and average collision rates for each of these traps. We have also included the results for a general isotropic power law trap, the potential of which is described by
\begin{equation}
    {\cal U}_{\mathrm{PL}}(\mathbf{r}) = {\cal U}_0 \left(\frac{\mathbf{r}}{r_e}\right)^{3/\gamma}, \label{eq:powerlaw}
\end{equation}
where the trap has a characteristic trap size $r_e$ and a trap strength of ${\cal U}_0$. The parameter, $\bar{v}$, in
 \autoref{tab:collisionrates} is the thermally averaged atomic speed and is given by
\begin{equation*}
    \bar{v} = \sqrt{\frac{8 k_B T}{\pi m}}.
\end{equation*}

\begin{table}
\hspace{-16em}
\myfloatalign
\begin{tabularx}{1.35\textwidth}{|l|c|c|c|} \toprule
\tableheadline{Trapping Potential} & \tableheadline{Trap Power} & \tableheadline{Effective Volume, $V_e$} & \tableheadline{ Collision Rate, ${\tau_c}^{-1}$} \\ \midrule
Homogeneous Gas & $\infty$? &  V & $\frac{1}{2^{1/2}}n_0\bar{v}\sigma$ \\
\midrule
Spherical Quadrupole & 1 & $256\pi\left(\frac{ k_B T}{g_s \mu_B B_z'}\right)^3$ & $\frac{1}{2^{7/2}}n_0\bar{v}\sigma$ \\
\midrule
Ioffe Pritchard & 2 & $\frac{8}{\sqrt{B''}B_\rho''}\left(\frac{ \pi k_B T}{g_s \mu_B}\right)^{3/2}$ & $\frac{1}{2^2}n_0\bar{v}\sigma$ \\
\midrule
Isotropic Power Law & $3/\gamma$ & $\frac{4}{3}\pi{r_e}^3\Gamma\left[\gamma+1\right]\left(\frac{k_B T}{{\cal U}_0}\right)^{\gamma}$ & $\frac{1}{2^{\gamma+0.5}}n_0\bar{v}\sigma$\\
\bottomrule
\end{tabularx}
\caption[Collision rates for different trapping potentials.]{Collision rates for different trapping potentials.}  
\label{tab:collisionrates}
\end{table}

We can make a few interesting observations from these calculations. The most interesting is that the collision rate in the trapped gases \emph{increases} as the temperature \emph{decreases}, which is the converse to the homogeneous gas\footnote{This relationship between collision rate and temperature for trapped gases is what gives rise to the "runaway" evaporation observed during atom cooling experiments.}. 
This is because for a trapped gas the central density increases at a rate greater than the decrease of the average thermal velocity, $\bar{v}$. 
For the homogeneous gas the density remains constant as the gas cools, thus the decrease in the velocity of the atoms results in an overall decrease in the collision rate.
We can also see that if we can hold the trap strength and effective trap size constant the central density of the trap will increase as the power of the trap decrease (or as $\gamma$ increases). 
This is often spoken about in terms of "tightness" and is a strong motivator behind the use of the quadrupole trap.
In section \ref{sec:evaporation} we will show that it is the tightness of a trap that determines it's efficiency in evaporative cooling.

%----------------------------------------------------------------------------------------

\section{Mean free path}
\label{sec:mfp}

Knudsen number, collisionless regime?