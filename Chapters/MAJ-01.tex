% Chapter X

\chapter{Majorana Interulde} % Chapter title

\label{ch:majinter} % For referencing the chapter elsewhere, use \autoref{ch:dsmcehr} 

%-----------------------------------------------------------------------------------------

General scope for the problem.

Inadequacies in alternative approaches.

Majorana's approach does not require the extension into the imaginary time domain nor does he make dodgey mistakes in an attempt to engineer the solution.

%-----------------------------------------------------------------------------------------

\section{Derivation of the Differential Equation}

To illustrate the theory of an avoided crossing we will consider the case of a static 2-level atom in a dynamic field. 
The interaction Hamiltonian for a magnetic dipole in a magnetic field is $\widehat{H}_{B} = -\hat{\boldsymbol{\mu}} \cdot \boldsymbol{B}$, written in matrix form is
\begin{equation}
	\widehat{H}_{B} = \hbar \gamma \begin{bmatrix} B_{z} & B_{x} - i B_{y}\\
											B_{x} + i B_{y} & -B_{z} \end{bmatrix}, \label{Eq:MagHam}
\end{equation}
where $\gamma = g_{F}\mu_{B}M_{F} / \hbar$ is the gyromagnetic ratio and represents the proportionality between the spin and magnetic moment, $\hat{\boldsymbol{\mu}}$, operators. 
If we choose the ``spin up,'' $\vert\uparrow\rangle$ and ``spin down,'' $\vert \downarrow\rangle$ states as our basis then, the diagonal elements are the energies of our basis (or diabatic) states and the off diagonal elements represent the coupling between states. 
The eigenvalues for the hamiltonian in \eqref{Eq:MagHam} are 
\begin{equation}
	E_{\pm} = \pm \hbar \gamma \sqrt{{B_{x}}^{2} + {B_{y}}^{2} + {B_{z}}^{2}},
\end{equation}
and correspond to the states perfectly aligned (and anti-aligned) with the local magnetic field. 
We can see the effect of the coupling between states in these eigen-energies through the appearance of the term, $\left\vert\Delta\right\vert^{2} = {B_{x}}^{2} + {B_{y}}^{2}$. 
More specifically consider the two situations shown in figure \ref{Fig:AvoidedCrossing}. 
When there is no coupling we get the grey curves and when there is coupling we get the coloured curves \textcolor{red}{(explain)}. 
If we consider a time-dependant magnetic field in which the coupling is constant and the diagonal field changes from large and negative to large positive very slowly (or \emph{adiabatically}) then the system should follow the one of the curves. 
If it goes quickly then it might jump across the gap \textcolor{red}{(this is rubbish)}.

The time-dependant magnetic field discussed in Majorana's paper \cite{Majorana1932} is ${\mathbf B} = \left(A, 0 , -Ct\right)$ and so the corresponding coupled time-dependant Schr\"odinger equations for a spin half particle with $\vert\psi(t)\rangle = c_{1}(t) \vert \downarrow\rangle + c_{2}(t) \vert \uparrow \rangle$, are
\begin{subequations}
\begin{align}
	\dot{c_{1}} &= - \gamma i \left(-Ctc_{1} + Ac_{2}\right),\\
	\dot{c_{2}} &= - \gamma i \left(A c_{1} + Ct c_{2}\right).
\end{align}
\end{subequations}
Using the dimensionless time
\begin{equation}
	\tau = \sqrt{\frac{\gamma \, C}{2}} \, t,
\end{equation}
and the numerical quantity
\begin{equation}
	k = \frac{2\gamma A^{2}}{C},
\end{equation}
which describes the ratio of the atom precession frequency to the rotation frequency of the field direction, we can obtain the coupled differential equations
\begin{subequations} \label{eq:Majprob}
\begin{align}
	\frac{dc_{1}}{d\tau} &= - i \left(-2\tau c_{1} + \sqrt{k}c_{2}\right),\\
	\frac{dc_{2}}{d\tau} &= -i \left(\sqrt{k} c_{1} + 2\tau c_{2}\right).
\end{align}
\end{subequations}
These equations can be further simplified through the application of the integration factor method \cite{Kreyszig2006}, by setting
\begin{equation}
	c_{1} = e^{i\tau^{2}}f, \quad \quad
	c_{2} = e^{-i\tau^{2}}g,
\end{equation}
from which it follows
\begin{subequations}
\begin{align}
	\frac{df}{d\tau} &= -i \sqrt{k} e^{-2i\tau^{2}}g,\\
	\frac{dg}{d\tau} &= -i \sqrt{k} e^{2i\tau^{2}}f. \label{Eq:dgdt}
\end{align}
\end{subequations}
Eliminating $g$, we obtain\footnote{The factor of $4i\tau$ was erroneously printed as $hi\tau$.}
\begin{equation}
	\frac{d^{2}f}{d\tau^{2}} + 4 i \tau \frac{df}{d\tau} + kf = 0. \label{Eq:fDiff}
\end{equation}

%-----------------------------------------------------------------------------------------

\section{Transforming into an Integral Equation}

When obtaining an asymptotic solution to a differential equation it is often preferable to investigate the asymptotic behaviour of an integral solution. 
It is not always possible to obtain an integral solution, but when possible it provides a global solution from which all asymptotic limits can be obtained, and hence connect solutions from one domain to another. 
It is with this motivation that we seek to find an integral representation of equation \eqref{Eq:fDiff}. 
To this end we will assume the solution can be written as an integral of the product between a given kernel $K(s,\tau)$ and some unknown function\cite{White2005} $\chi(s)$
\begin{equation}
	f(\tau) = \int_{C} K(s,\tau) \chi(s) \, ds,
\end{equation}
where $C$ is an arbitrary contour in the complex plane and $K$ and $\chi$ are \emph{complex functions} of the \emph{complex variable} $s$. 
We will use the Fourier Laplace kernel, $K(s,\tau) = e^{s\tau}$ so that the differential equation \eqref{Eq:fDiff} becomes
\begin{equation}
	\int_{C}\chi(s) \left(s^{2}+k\right)e^{s\tau}\, ds + 4i\int_{C}\chi(s)s\tau e^{s\tau} \, ds = 0.
\end{equation}
If we now integrate the second term by parts we have
\begin{equation}
	\left.\int_{C}\left[\left(s^{2}+k-4i\right)\chi(s) - 4is \chi'(s)\right] e^{s\tau}\, ds + s\chi(s)e^{s\tau}\right\vert_{C} = 0.
\end{equation}
Choosing the contour, $C$, such that the boundary term disappears we are left with the differential equation
\begin{equation}
	\left(s^{2}+k-4i\right)\chi(s) - 4is \chi'(s) = 0,
\end{equation}
which is separable and has solution the solution
\begin{equation}
	\chi(s) = As^{(k/4i) - 1}e^{s^{2}/8i},
\end{equation}
so long as $\log_{e}s$ assumes its principle value. 
Here $A$ is a constant of normalisation, to be determined in section \ref{Sec:Asymp}. 
Thus we have 
\begin{equation}
	f(\tau) = A\int_{C}s^{(k/4i)-1}e^{(s^{2}/8i)+s\tau}\, ds, \label{Eq:AsymInt}
\end{equation}
with the boundary condition
\begin{equation}
	\left.s^{k/4i}e^{(s^{2}/8i)+s\tau}\right\vert_{C}=0.
\end{equation}
\textcolor{red}{(for example consider the rays where $\text{Arg} \,s = -\frac{\pi}{4}, \, \frac{3\pi}{4}$.)}

%-----------------------------------------------------------------------------------------

\section{Asymptotic Solution of the Integral} \label{Sec:Asymp}

The simplest technique for obtaining the asymptotic behaviour as $\tau \to +\infty$ of integrals in which the large parameter $\tau$ appears in an exponential
\begin{equation}
	I(\tau) = \int_{a}^{b} h(s) e^{\tau\phi(s)} \, ds,
\end{equation}
is Laplace's method \cite{Bender1999} where we assume $h(s)$ and $\phi(s)$ are \emph{real} and \emph{continuous}. 
The crux of Laplace's method is the idea that if the real continuous function $\phi(s)$ has its \emph{maximum} on the interval $a \le s \le b$ at $t = c$ and if $h(c) \ne 0$, then it is only the \emph{immediate neighbourhood} of $t=c$ that contributes to the full asymptotic expansion of $I(\tau)$ for large $\tau$. 
That is we may approximate the integral $I(\tau)$ by $I(\tau; \epsilon)$ where
\begin{align}
	I(\tau;\epsilon) &= \int_{c-\epsilon}^{c+\epsilon}h(s) e^{\tau\phi(s)}\, ds,\\
	&\approx \int_{c-\epsilon}^{c+\epsilon}\left[h(c) + \left(s-c\right)h'(c) + {\cal O}(s^{2})\right]e^{\tau\left[\phi(c)+\left(s-c\right)\phi'(c) + \frac{1}{2}\left(s-c\right)^{2}\phi''(c)+{\cal O}(s^{3})\right]}\, ds,\\
	&\approx h(c)e^{\tau\phi(c)}\int_{-\infty}^{\infty} e^{\tau\left(s-c\right)^{2}\phi''(s)/2}\, ds, \quad \text{as }\tau \to + \infty. \label{Eq:Laplace}
\end{align}
However we notice straight away that our function $h(s) = s^{(k/4i)-1}e^{s^{2}/8i}$ is not real, nor is the variable of integration, $s$, thus Laplace's method is not directly applicable, we will instead, employ a technique known as the method of steepest descents \cite{Bender1999}. 
The method of steepest descents is a technique for finding the asymptotic behaviour of integrals of the form
\begin{equation}
	I(\tau) = \int_{C}h(s) e^{\tau \rho(s)} \, ds, \label{Eq:SteepestDescent}
\end{equation}
as $\tau \to +\infty$, where $C$ is an integration contour in the complex $s$ plane and $h(s)$ and $\rho(s)$ are \emph{analytic functions} of $s$. 
Here the idea is to use the analyticity of the integrand to justify deforming the contour $C$ to a new contour $C'$ on which $\rho(s)$ has a constant imaginary part. 
Once this has been done, $I(s)$ may be evaluated asymptotically as $\tau \to +\infty$ using Laplace's method. 
To see why, observe that on the contour $C'$ we may write $\rho(s) = \phi(s) + i \psi$, where $\psi$ is a real constant and $\phi(s)$ is a real function. 
Thus, $I(s)$ in \eqref{Eq:SteepestDescent} takes the form
\begin{equation}
	I(s) = e^{is\psi} \int_{C'}h(s) e^{\tau\phi(s)}\, ds.
\end{equation}
In our case we cannot simply use $\rho(s) = s$ since it's maximum is infinite and we expect the integral to be convergent, we instead have to consider the entire exponent $\rho(s) = s^{2}/8i+s\tau$. 
If we plan on using Laplace's method to approximate the integral along the deformed contour then we require that the new contour passes through the stationary point of $\rho(s)$. 
The stationary point of $\rho(s)$ will be at $ s_{\text{st.pt.}} = -4i\tau$. Now we let $s = p + iq$ so that  
\begin{equation}
	\rho(s) = \rho(p,q) = \left[\frac{1}{4}pq + \tau p\right] + i \left[ \frac{1}{8}q^{2} - \frac{1}{8}p^{2} + \tau q\right],
\end{equation}
thus the imaginary part of the exponent at the stationary point is $\Im(\rho(s_{\text{st.pt}})) = \Im(\rho(0,-4\tau)) = -2\tau^{2}$. 
The contours along which the imaginary part of $\rho(s)$ is constant and equal to $\text{Im}(\rho(s_{\text{st.pt.}}))$ are thus given by
\begin{align}
	-2\tau^{2} &= \frac{1}{8}q^{2} + q\tau - \frac{1}{8}p^{2},\\
	\Rightarrow q &= -4\tau \pm p.
\end{align}
These two contours are shown in figure \ref{Fig:ContourPlots}. 
Since we require that the stationary point $s_{\text{st.pt.}}=-4i\tau$ be a maximum we will choose the contour $y=-4\tau-p$ so that we now have
\begin{equation}
	s = -4i \tau + \left(1 - i\right)p, \quad ds = \left(1 - i\right) dp.
\end{equation}
Making this substitution the integral \eqref{Eq:AsymInt} becomes
\begin{equation}
	f(\tau) =A\left(1-i\right)e^{-2i\tau^{2}}\int_{-\infty}^{\infty}\left(-4i\tau + \left(1-i\right)p\right)^{(k/4i)-1}e^{-p^{2}/4}\, dp.
\end{equation}
While $\tau < 0$ we can apply equation \eqref{Eq:Laplace} directly to find
\begin{align}
	f(\tau) &\approx A\left(1-i\right)e^{-2i\tau^{2}}\left(-4i \tau\right)^{k/4i-1}\int_{-\infty}^{\infty}e^{-p^{2}/4}\, dp, \quad \text{as } \tau \to -\infty\\
	&= 2A\left(1-i\right)\sqrt{\pi}e^{-2i\tau^{2}}\left(-4i \tau\right)^{k/4i-1}, \quad \text{as } \tau \to -\infty.
\end{align}
In the limit $\tau \to -\infty$ we find $f(\tau) = 0$ and using equation \eqref{Eq:dgdt} we have 
\begin{align}
	g(\tau) &= \int_{-\infty}^{\tau}2kAi\left(1-i\right)\sqrt{\pi}\left(-4i \tau' \right)^{k/4i-1}\, d\tau', \quad \text{as } \tau \to -\infty\\
	&= 2A\left(1+i\right)\sqrt{\pi}\left(-4 \tau' \right)^{k/4i}e^{k\pi/8} , \quad \text{as } \tau \to -\infty.
\end{align}
We can see that if $\left\vert g\right\vert^{2} = 1$ as $\tau \to - \infty$ we require $A = \sqrt{k}e^{-k\pi/8}/(2(1+i)\sqrt{\pi})$.
When $\tau >0$ the integration contour will have to cross a branch cut along the negative real axis caused by the fractional powers of $s$. 
In this case we will follow the path indicated in figure \ref{Fig:IntegrationPaths}. 
So now let
\begin{align}
	f(\tau) &= \int_{\gamma_{1}} + \int_{\gamma_{2}} + \int_{\gamma_{3}} + \int_{\gamma_{4}} + \int_{\gamma_{5}},\\
	&= f_{1} + f_{2} + f_{3} + f_{4} + f_{5} .
\end{align}
Can show $f_{1}$ and $f_{3}$ tend to zero (at least in the limit $\tau \to 0$). Along $\gamma_{2}$ we let $s = pe^{i \pi}$ so that $ds = e^{i \pi}dp$ and
\begin{equation}
	f_{2} = \frac{\sqrt{k} e^{-k\pi/8}}{2\left(1+i\right)\sqrt{\pi}}\int_{-4\tau}^{\epsilon} p^{(k/4i)-1}e^{k\pi/4}e^{(p^{2}/8i) -p\tau}\, dp.
\end{equation}
Similarly along $\gamma_{4}$ we let $s = pe^{-i\pi}$ so that $ds = e^{-i\pi}dp$ and
\begin{equation}
	f_{4} = -\frac{\sqrt{k} e^{-k\pi/8}}{2\left(1+i\right)\sqrt{\pi}}\int_{-4\tau}^{\epsilon} p^{(k/4i)-1}e^{-k\pi/4}e^{(p^{2}/8i) -p\tau}\, dp.
\end{equation}
Now adding $f_{2}$ and $f_{4}$ together we have
\begin{align}
	f_{2} + f_{4} &= \frac{\sqrt{k} e^{-k\pi/8}}{2\left(1+i\right)\sqrt{\pi}}\int_{-4\tau}^{\epsilon} p^{(k/4i)-1}e^{(p^{2}/8i) -p\tau}\left(e^{k\pi/4}-e^{-k\pi/4}\right)\, dp,\\
	&= \frac{\sqrt{k} e^{-k\pi/8}}{\left(1+i\right)\sqrt{\pi}}\sinh\left(\frac{k\pi}{4}\right)\int_{-4\tau}^{\epsilon} p^{(k/4i)-1}e^{(p^{2}/8i) -p\tau}\, dp,\\
	&= \frac{\sqrt{k} e^{-k\pi/8}}{\left(1+i\right)\sqrt{\pi}}\sinh\left(\frac{k\pi}{4}\right)\int_{-4\tau}^{\epsilon} p^{(k/4i)-1}e^{-p\tau}\left(1 + \frac{p^{2}}{8i} + {\cal O}(p^{4})\right)\, dp.
\end{align}
Looking at the above equation it is clear that if we make the substitution $p\tau = p'$ so that $dp = \tau^{-1}dp'$ we have will have (to the highest order in $\tau$) the integral form of the gamma function so that
\begin{equation}
	f_{2}+f_{4} = \frac{\sqrt{k} e^{-k\pi/8}}{\left(1+i\right)\sqrt{\pi}}\sinh\left(\frac{k\pi}{4}\right)\tau^{-k/4i}\Gamma\left(\frac{k}{4i}\right).
\end{equation}
We will also have the contribution from the saddle point and will be the same as for the $\tau < 0$ case so 
\begin{equation}
	f_{5} = -i\sqrt{k}e^{k\pi/8-2i\tau^{2}}\left(-4i \tau\right)^{-1-k/4i}, \quad \text{as } \tau \to \infty.
\end{equation}
The $\gamma_{1}$ contour will follow the same derivation as the $\gamma_{5}$ except that it's maximum is at $x=-4\tau$ so the Taylor series expansion must be made about this maximum. 
Thus we have
\begin{equation}
	f_{1} = -i\sqrt{k}e^{k\pi/8-2i\tau^{2}}\left(-4 \tau\right)^{-1-k/4i}, \quad \text{as } \tau \to \infty,
\end{equation}
which will decay exponentially as $\tau \to \infty$, faster than $f_{5}$.
Adding all of our results together we find\footnote{The $\tau^{-k/4i}$ was erroneously printed as $e^{-k/4i}$.} in the limit as $\tau \to \infty$
\begin{equation}
	f(\tau) =  \frac{\sqrt{k} e^{-k\pi/8}}{\left(1+i\right)\sqrt{\pi}}\sinh\left(\frac{k\pi}{4}\right)\tau^{-k/4i}\Gamma\left(\frac{k}{4i}\right),
\end{equation}
and 
\begin{equation}
	g(\tau) = (4\tau)^{k/4i}e^{-k\pi/4}.
\end{equation}

%-----------------------------------------------------------------------------------------

\section{Conclusion}

The probability of the spin adiabatically flipping will be given by $\left\vert g \right\vert^{2} = e^{-k\pi/2}$ as $\tau \to \infty$. \textcolor{red}{(Explain the changing sign of the magnetic field swaps the basis vectors.)}

%----------------------------------------------------------------------------------------

\section{Majornan Spin Flips}
\begin{equation}
    \frac{\partial}{\partial t} \label{eq:Majprob}
\end{equation}

Talk about Majorana problem, history, derive formula etc

%----------------------------------------------------------------------------------------

\section{Landau Zener Formula}

Content

\section{Loss Rates 'n' Stuff}

derive the loss rate formulae used in all the literature.